\documentclass[12pt, leqno]{article} %% use to set typesize
\usepackage{fancyhdr}
\usepackage[sort&compress]{natbib}
\usepackage[letterpaper=true,colorlinks=true,linkcolor=black]{hyperref}

\usepackage{amsfonts}
\usepackage{amsmath}
\usepackage{amssymb}
\usepackage{color}
\usepackage{tikz}
\usepackage{pgfplots}
\usepackage{listings}
%\usepackage{courier}
%\usepackage[utf8]{inputenc}
%\usepackage[russian]{babel}

\lstset{
  numbers=left,
  basicstyle=\ttfamily\footnotesize,
  numberstyle=\tiny\color{gray},
  stepnumber=1,
  numbersep=10pt,
}

\newcommand{\iu}{\ensuremath{\mathrm{i}}}
\newcommand{\bbR}{\mathbb{R}}
\newcommand{\bbC}{\mathbb{C}}
\newcommand{\calV}{\mathcal{V}}
\newcommand{\calW}{\mathcal{W}}
\newcommand{\macheps}{\epsilon_{\mathrm{mach}}}
\newcommand{\matlab}{\textsc{Matlab}}

\newcommand{\ddiag}{\operatorname{diag}}
\newcommand{\fl}{\operatorname{fl}}
\newcommand{\nnz}{\operatorname{nnz}}
\newcommand{\tr}{\operatorname{tr}}
\renewcommand{\vec}{\operatorname{vec}}

\newcommand{\vertiii}[1]{{\left\vert\kern-0.25ex\left\vert\kern-0.25ex\left\vert #1
    \right\vert\kern-0.25ex\right\vert\kern-0.25ex\right\vert}}
\newcommand{\ip}[2]{\langle #1, #2 \rangle}
\newcommand{\ipx}[2]{\left\langle #1, #2 \right\rangle}
\newcommand{\order}[1]{O( #1 )}

\newcommand{\kron}{\otimes}


\newcommand{\hdr}[1]{
  \pagestyle{fancy}
  \lhead{Bindel, Summer 2018}
  \rhead{Numerics for Data Science}
  \fancyfoot{}
  \begin{center}
    {\large{\bf #1}}
  \end{center}
  \lstset{language=matlab,columns=flexible}  
}


\begin{document}
\hdr{2018-06-14}

%\section{Linear and nonlinear least squares}




\section{Iterative methods}

We started this lecture with a discussion of {\em direct} methods
for solving least squares problems based on matrix factorizations.
These methods have a well-understood running time, and they produce
a solution that is accurate except for roundoff effects.  For larger
or more complicated problems, though, we need to turn to
{\em iterative} methods that produce a series of approximations
to the problem solution.

We will spend the next three lectures talking about gradient and
stochastic gradient methods, Newton and Gauss-Newton, and (block)
coordinate descent.  We will see additional solver ideas as we move
through the semester, but these are nicely prototypical examples that
illustrate two running themes in the design of numerical methods for
optimization.

\paragraph{Fixed point iterations}
All our nonlinear solvers (and some of our linear solvers) will be
{\em iterative}.  We can write most as {\em fixed point iterations}
\begin{equation}
  x^{k+1} = G(x^k), \label{eq:fixed-point}
\end{equation}
which we hope will converge to a fixed point, i.e. $x^* = G(x^*)$.
We often approach convergence analysis through the
{\em error iteration} relating the error $e^k = x^k-x^*$ at
successive steps:
\begin{equation}
  e^{k+1} = G(x^* + e^k)-G(x^*).
\end{equation}

\paragraph{Model-based methods}
Most nonquadratic problems are too hard to solve directly.  On the other
hand, we can {\em model} hard nonquadratic problems by simpler (possibly
linear) problems as a way of building iterative solvers.  The most
common tactic --- but not the only one! --- is to approximate the
nonlinear function by a linear or quadratic function and apply all the
things we know about linear algebra.  We will return to this idea in
when we discuss Newton-type methods for optimization.

\section{Gradient descent}

One very simple iteration is {\em steepest descent}
or {\em gradient descent}:
\begin{equation} \label{iter:gd}
  x^{k+1} = x^k + \alpha_k p^k
\end{equation}
where the step direction $p^k = -\nabla \phi(x^k)$ is the negative
gradient and $\alpha_k$ is the {\em step size}, which we can
choose adaptively or by some fixed schedule.

To understand the convergence of this method, consider gradient
descent with a fixed step size $\alpha$ for the quadratic model problem
\[
  \phi(x) = \frac{1}{2} x^T A x + b^T x + c
\]
where $A$ is symmetric positive definite.  
We have computed the gradient for a quadratic before:
\[
  \nabla \phi(x) = Ax + b,
\]
which gives us the iteration equation
\[
  x_{k+1} = x_k - \alpha (A x_k + b).
\]
Subtracting the fixed point equation
\[
  x_* = x_* - \alpha (A x_* + b)
\]
yields the error iteration
\[
  e_{k+1} = (I-\alpha A) e_k.
\]
If $\{ \lambda_j \}$ are the eigenvalues of $A$, then the
eigenvalues of $I-\alpha A$ are $\{ 1-\alpha \lambda_j \}$.
The spectral radius of the iteration matrix is thus
\[
  \min \{ |1-\alpha \lambda_j| \}_j =
  \min \left( |1-\alpha \lambda_{\min}|, |1-\alpha \lambda_{\max}| \right).
\]
The iteration converges provided $\alpha < 1/\lambda_{\max}$, and the
optimal $\alpha$ is
\[
  \alpha_* = \frac{2}{\lambda_{\min} + \lambda_{\max}},
\]
which leads to the spectral radius
\[
  1 - \frac{2 \lambda_{\min}}{\lambda_{\min} + \lambda_{\max}} =
  1 - \frac{2}{1 + \kappa(A)}
\]
where $\kappa(A) = \lambda_{\max}/\lambda_{\min}$ is the condition
number for the (symmetric positive definite) matrix $A$.  If $A$
is ill-conditioned, then, we are forced to take very small steps
to guarantee convergence, and convergence may be
heart breakingly slow.  We will get to the minimum in the long run
--- but, then again, in the long run we all die.

How steepest descent behaves on a quadratic model
is how it behaves generally: if $x_*$ is a
strong local minimizer of some general nonlinear $\phi$, then gradient
descent with a small enough step size will converge locally to
$x_*$.  But if $H_{\phi}(x_*)$ is ill-conditioned, then one has to
take small steps, and convergence can be quite slow.
Somewhat surprisingly, sometimes we {\em want} this slow convergence;
the {\em Landweber iteration} is steepest descent iteration applied to
linear least squares problems is sometimes used to solve very
ill-conditioned least squares problems, e.g.~in image reconstruction.
We will return to why this is a good idea in a later lecture,
when we discuss {\em regularization}.

Not all problems are terrible ill-conditioned, and so in many cases
simple gradient descent algorithms can work quite well.  For
ill-conditioned problems, though, we would like to change something
about the algorithm.  One approach is to keep the gradient descent
direction and adapt the step size in a clever way; the
Barzelei-Borwein (BB) method and related approaches follow this
approach.  These remarkable methods deserve to be better known,
but in the interest of time, we will move on with only a name drop.

\section{Gradient descent with errors}

Before we turn to stochastic gradient descent, let us instead look at
how to analyze {\em gradient descent with errors}.  In particular,
consider the iteration
\[
  x^{k+1} = x^k - \alpha_k p^k
\]
where
\[
  p^k = \nabla \phi(x^k) + u^k
\]
for some error $u^k$ that is ``small.''  As before, let's keep things
simple by looking how this iteration behaves for a quadratic model
problem with a fixed step size, i.e.
\[
  x^{k+1} = x^k - \alpha (Ax^k + b + u^k).
\]
Subtracting $x^*$ from both sides gives the error iteration
\[
  e^{k+1} = (I-\alpha A) e^k - \alpha u^k.
\]  
A little mumbling over the iteration gives us
\[
  e^{k+1} = (I-\alpha A)^{k+1} e^0 - \alpha \sum_{j=0}^k (I-\alpha A)^{k-j} u^j.
\]

In order to analyze the second term in this iteration, we need some
additional sort of control.  In the simplest case, that control might
be deterministic.  For example, if we can guarantee that
$\|u^k\| \leq C \gamma^{-k}$, then we have the bound
\[
  \left\| \sum_{j=0}^k  (I-\alpha A)^{k-j} u_j \right\| \leq
  C \gamma^{-k} \sum_{j=0}^k \left( \gamma \|(I-\alpha A)\|\right)^{k-j} \leq
  \frac{C \gamma^{-k-1}}{\|I-\alpha A\|}.
\]
Hence, we can make the iteration converge with inaccurate gradients,
as long as the accuracy improves sufficiently quickly with time.

Another approach might be to assume that the errors are independent
errors with mean zero and covariance matrix $\Sigma$.  This is
essentially what will happen in the {\em stochastic gradient descent}
method, to be introduced soon.  In this case, we can interpret the iteration
as a Markov chain, where the mean of the stationary distribution is
the true solution $x^*$ and the covariance matrix is given by
\[
  \Sigma^\infty = \sum_{j=0}^\infty (I-\alpha A)^j \Sigma (I-\alpha A)^j,
\]
which is also the solution to the algebraic Riccati equation\footnote{%
Do I expect you to know about algebraic Riccati equations?  Of course
not.  But I think it is wonderful that there is a connection between
the analysis of gradient descent and a nonlinear matrix equation that
usually arises in control theory, so I will drop the name and let you
read more if you care to do so.
}
\[
  \Sigma^\infty = (I-\alpha) (\Sigma^\infty + \Sigma) (I-\alpha).
\]
This is not completely trivial to analyze, but at least we know the
sum converges, assuming that the step size is chosen so that the
solution without error converges.
Of course, one might find this not particularly satisfactory, since
we would ideally like the error in a solution algorithm to go to zero;
an estimator with non-vanishing variance may not cut it.  In order to
achieve this, we need to understand what happens with variable
step sizes.

%\section{Stochastic gradient descent}

%\section{Scaling and Newton}

%\section{Block coordinate descent}

\end{document}
