\documentclass[12pt, leqno]{article} %% use to set typesize
\usepackage{fancyhdr}
\usepackage[sort&compress]{natbib}
\usepackage[letterpaper=true,colorlinks=true,linkcolor=black]{hyperref}

\usepackage{amsfonts}
\usepackage{amsmath}
\usepackage{amssymb}
\usepackage{color}
\usepackage{tikz}
\usepackage{pgfplots}
\usepackage{listings}
%\usepackage{courier}
%\usepackage[utf8]{inputenc}
%\usepackage[russian]{babel}

\lstset{
  numbers=left,
  basicstyle=\ttfamily\footnotesize,
  numberstyle=\tiny\color{gray},
  stepnumber=1,
  numbersep=10pt,
}

\newcommand{\iu}{\ensuremath{\mathrm{i}}}
\newcommand{\bbR}{\mathbb{R}}
\newcommand{\bbC}{\mathbb{C}}
\newcommand{\calV}{\mathcal{V}}
\newcommand{\calW}{\mathcal{W}}
\newcommand{\macheps}{\epsilon_{\mathrm{mach}}}
\newcommand{\matlab}{\textsc{Matlab}}

\newcommand{\ddiag}{\operatorname{diag}}
\newcommand{\fl}{\operatorname{fl}}
\newcommand{\nnz}{\operatorname{nnz}}
\newcommand{\tr}{\operatorname{tr}}
\renewcommand{\vec}{\operatorname{vec}}

\newcommand{\vertiii}[1]{{\left\vert\kern-0.25ex\left\vert\kern-0.25ex\left\vert #1
    \right\vert\kern-0.25ex\right\vert\kern-0.25ex\right\vert}}
\newcommand{\ip}[2]{\langle #1, #2 \rangle}
\newcommand{\ipx}[2]{\left\langle #1, #2 \right\rangle}
\newcommand{\order}[1]{O( #1 )}

\newcommand{\kron}{\otimes}


\newcommand{\hdr}[1]{
  \pagestyle{fancy}
  \lhead{Bindel, Summer 2018}
  \rhead{Numerics for Data Science}
  \fancyfoot{}
  \begin{center}
    {\large{\bf #1}}
  \end{center}
  \lstset{language=matlab,columns=flexible}  
}


\begin{document}
\hdr{2018-06-15}

\section{Linear and nolinear least squares}

Another approach might be to assume that the errors are independent
errors with mean zero and covariance matrix $\Sigma$.  This is
essentially what will happen in the {\em stochastic gradient descent}
method, to be introduced soon.  In this case, we can interpret the iteration
as a Markov chain, where the mean of the stationary distribution is
the true solution $x^*$ and the covariance matrix is given by
\[
  \Sigma^\infty = \sum_{j=0}^\infty (I-\alpha A)^j \Sigma (I-\alpha A)^j,
\]
which is also the solution to the algebraic Riccati equation\footnote{%
Do I expect you to know about algebraic Riccati equations?  Of course
not.  But I think it is wonderful that there is a connection between
the analysis of gradient descent and a nonlinear matrix equation that
usually arises in control theory, so I will drop the name and let you
read more if you care to do so.
}
\[
  \Sigma^\infty = (I-\alpha) (\Sigma^\infty + \Sigma) (I-\alpha).
\]
This is not completely trivial to analyze, but at least we know the
sum converges, assuming that the step size is chosen so that the
solution without error converges.
Of course, one might find this not particularly satisfactory, since
we would ideally like the error in a solution algorithm to go to zero;
an estimator with non-vanishing variance may not cut it.  In order to
achieve this, we need to understand what happens with variable
step sizes.

\section{Stochastic gradient descent}

\section{Scaling and Newton}

\section{Block coordinate descent}

\end{document}
