\documentclass[12pt, leqno]{article} %% use to set typesize
\usepackage{fancyhdr}
\usepackage[sort&compress]{natbib}
\usepackage[letterpaper=true,colorlinks=true,linkcolor=black]{hyperref}

\usepackage{amsfonts}
\usepackage{amsmath}
\usepackage{amssymb}
\usepackage{color}
\usepackage{tikz}
\usepackage{pgfplots}
\usepackage{listings}
%\usepackage{courier}
%\usepackage[utf8]{inputenc}
%\usepackage[russian]{babel}

\lstset{
  numbers=left,
  basicstyle=\ttfamily\footnotesize,
  numberstyle=\tiny\color{gray},
  stepnumber=1,
  numbersep=10pt,
}

\newcommand{\iu}{\ensuremath{\mathrm{i}}}
\newcommand{\bbR}{\mathbb{R}}
\newcommand{\bbC}{\mathbb{C}}
\newcommand{\calV}{\mathcal{V}}
\newcommand{\calW}{\mathcal{W}}
\newcommand{\macheps}{\epsilon_{\mathrm{mach}}}
\newcommand{\matlab}{\textsc{Matlab}}

\newcommand{\ddiag}{\operatorname{diag}}
\newcommand{\fl}{\operatorname{fl}}
\newcommand{\nnz}{\operatorname{nnz}}
\newcommand{\tr}{\operatorname{tr}}
\renewcommand{\vec}{\operatorname{vec}}

\newcommand{\vertiii}[1]{{\left\vert\kern-0.25ex\left\vert\kern-0.25ex\left\vert #1
    \right\vert\kern-0.25ex\right\vert\kern-0.25ex\right\vert}}
\newcommand{\ip}[2]{\langle #1, #2 \rangle}
\newcommand{\ipx}[2]{\left\langle #1, #2 \right\rangle}
\newcommand{\order}[1]{O( #1 )}

\newcommand{\kron}{\otimes}


\newcommand{\hdr}[1]{
  \pagestyle{fancy}
  \lhead{Bindel, Summer 2018}
  \rhead{Numerics for Data Science}
  \fancyfoot{}
  \begin{center}
    {\large{\bf #1}}
  \end{center}
  \lstset{language=matlab,columns=flexible}  
}


\begin{document}
\hdr{2018-07-03}

\section{Graphs and linear algebra}

Formally, an unweighted graph is
$\calG = (\calV, \calE)$,
where $\calE \subset \calV \times \calV$.
Informally, $\calV$ consists of things we want to model and $\calE$
represents the relations between them.
It is a very flexible representation: we use graphs to represent
friendships between people, wires between routers, citations between
papers, links between objects in a data structure, and many other things.
When the bare topology of the relationships does not provide enough
modeling power, we might also consider including functions on
$\mathcal{V}$ or $\mathcal{E}$ corresponding to different attributes.
The most common case is a scalar {\em weight} function
assigned to each edge that corresponds to the importance of the
relation: in a social network, for example, maybe a close and
active friendship has more weight than a casual acquaintance.

\section{Adjacency and degrees}

\section{Random walks and normalized adjacency}

\section{PageRank}

\section{Discrete gradients and the combinatorial Laplacian}

\section{Discrete sums and the signless Laplacian}

\section{Modularity matrices}

\section{Motif adjacency}


\end{document}
