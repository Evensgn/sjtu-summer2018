\documentclass[12pt, leqno]{article} %% use to set typesize
\usepackage{fancyhdr}
\usepackage[sort&compress]{natbib}
\usepackage[letterpaper=true,colorlinks=true,linkcolor=black]{hyperref}

\usepackage{amsfonts}
\usepackage{amsmath}
\usepackage{amssymb}
\usepackage{color}
\usepackage{tikz}
\usepackage{pgfplots}
\usepackage{listings}
%\usepackage{courier}
%\usepackage[utf8]{inputenc}
%\usepackage[russian]{babel}

\lstset{
  numbers=left,
  basicstyle=\ttfamily\footnotesize,
  numberstyle=\tiny\color{gray},
  stepnumber=1,
  numbersep=10pt,
}

\newcommand{\iu}{\ensuremath{\mathrm{i}}}
\newcommand{\bbR}{\mathbb{R}}
\newcommand{\bbC}{\mathbb{C}}
\newcommand{\calV}{\mathcal{V}}
\newcommand{\calW}{\mathcal{W}}
\newcommand{\macheps}{\epsilon_{\mathrm{mach}}}
\newcommand{\matlab}{\textsc{Matlab}}

\newcommand{\ddiag}{\operatorname{diag}}
\newcommand{\fl}{\operatorname{fl}}
\newcommand{\nnz}{\operatorname{nnz}}
\newcommand{\tr}{\operatorname{tr}}
\renewcommand{\vec}{\operatorname{vec}}

\newcommand{\vertiii}[1]{{\left\vert\kern-0.25ex\left\vert\kern-0.25ex\left\vert #1
    \right\vert\kern-0.25ex\right\vert\kern-0.25ex\right\vert}}
\newcommand{\ip}[2]{\langle #1, #2 \rangle}
\newcommand{\ipx}[2]{\left\langle #1, #2 \right\rangle}
\newcommand{\order}[1]{O( #1 )}

\newcommand{\kron}{\otimes}


\newcommand{\hdr}[1]{
  \pagestyle{fancy}
  \lhead{Bindel, Summer 2018}
  \rhead{Numerics for Data Science}
  \fancyfoot{}
  \begin{center}
    {\large{\bf #1}}
  \end{center}
  \lstset{language=matlab,columns=flexible}  
}


\begin{document}
\hdr{2018-06-27}

% Kernels four ways: feature maps, basis functions, energies, covariances

\section{Kernels}

{\em Kernels} are key to many data analysis methods.  A kernel is a
function of two arguments, and often we think of $k(x,y)$ as a measure
of how similar or related the objects $x$ and $y$ are to each other.
But there are several stories we tell that describe different ways to
think about kernels.  We discuss four such approaches in this lecture:
\begin{enumerate}
\item
  Kernels come from composing linear methods (e.g.~linear regression)
  with {\em feature maps}, nonlinear functions that map from the
  original function domain (say $\bbR^n$) into space of much higher
  dimension ($\bbR^N$).  In this story, the kernel represents an inner product
  between feature vectors.
\item
  Kernels define {\em basis functions for approximation spaces}.  Unlike
  some schemes to approximate from a fixed function space
  (e.g.~low-degree polynomials), kernel methods are {\em adaptive},
  allowing more variation in regions of the function domain where we
  have more data.
\item
  Kernels are associated with a {\em quadratic form on a space of
  functions}, which we think of as ``energy.''  Sometimes, as in the
  case of cubic splines and thin plate splines, the energy and kernel methods
  for regression minimize this quadratic form subject to data
  constraints.  Thinking of kernel methods in this way gives us one
  framework for understanding the error in kernel methods.
\item
  Kernels are also used to represent the {\em covariance of random
  processes} in general, and Gaussian processes in particular.  In this
  view of kernel-based regression, we start with a prior distribution
  over functions, and then use Bayes rule to get a posterior
  distribution based on measurements of the function.
\end{enumerate}
In each case, using
a kernel leads us in the end to a linear algebra problem: all the
nonlinearity in a kernel method is summarized in the kernel construction.

These notes are strongly influenced by the Acta Numerica article
``\href{https://dx.doi.org/10.1017/S0962492906270016}{Kernel
  techniques: From machine learning to meshless methods}''
by Robert Schaback and Holger Wendland.  This survey article runs to
98 pages, but if you have the time and inclination to read more
deeply, I strongly recommend it s a starting point!

\subsection{Some common examples}

The choice of an appropriate kernel is the key to the success
or failure of a kernel method.  But we often lack the insight to make
an inspired choice of kernels, and so fall back on a handful of
standard families of kernels.  Let us mention a few
that are commonly used in function approximation on $\bbR^n$.

\paragraph*{Squared exponential}
The {\em squared exponential} kernel has the general form
\[
  k_{\mathrm{SE}}(x,y) =
    s^2 \exp\left( -\frac{1}{2} (x-y)^T P^{-1} (x-y) \right),
\]
for positive $s$ and positive definite $P$.  Most often we
choose $P = \ell^2 I$, i.e.
\[
  k_{\mathrm{SE}}(x,y) =
    s^2 \exp\left( -\frac{1}{2} \left(\frac{\|x-y\|}{\ell} \right)^2 \right).
\]
The {\em scale factor} $s$ and the {\em length scale} $\ell$ are
examples of kernel {\em hyper-parameters}.  In the case where we use a
single length scale parameter (rather than a more general $P$), the
squared exponential kernel is an example of
a {\em radial basis function}, i.e.~a kernel that depends only on
the distance between the two arguments.  In some corners of machine
learning, this kernel is referred to as ``the'' radial basis function,
but we will avoid this usage in deference to the many other useful
radial basis functions in the world.

\paragraph*{Exponential}
The (absolute) {\em exponential kernel} has the form
\[
  k_{\mathrm{exp}}(x,y) =
    s^2 \exp\left( -\frac{\|x-y\|}{\ell} \right)
\]
where $s$ and $\ell$ are again the scale factor and length scale
hyper-parameters.  As with the squared exponential kernel, there
is a more general form in which $\|x-y\|/\ell$ is replaced by
$\|x-y\|_{P^{-1}}$ for some positive definite $P$ matrix.
Where the squared exponential function is smooth,
the exponential kernel is only continuous --- it is not
differentiable.  This has important implications in modeling,
as the function approximations produced by kernel methods inherit the
smoothness of the kernel.  Hence, a smooth kernel (like the squared
exponential) is good for fitting smooth functions, while a
non-differentiable kernel (like the absolute exponential) may be a
better choice for fitting non-differentiable functions.

\paragraph*{Cubic splines and thin plate splines}
At the end of the last lecture, we saw that we can write
{\em cubic splines} for 1D function approximation 
in terms of the kernel $k(x,y) = |x-y|^3$.  For functions
on $\bbR^2$, the analogous choice is the {\em thin plate spline}
with kernel $r \log r$ for $r = \|x-y\|$.  Unlike the squared
exponential and exponential kernels, we express these kernels in terms
of the distance $r$ and not a scaled distance $r/\ell$, as
approximation methods using cubic and thin plate splines are
{\em scale invariant}: scaling the coordinate system does not change
the predictions.  Also unlike the squared
exponential and exponential kernels, these kernels {\em grow} with
increasing values of $r$, and so our intuition that kernels measure
``how similar things are'' founders a bit on these examples.  Other
ways of explaining kernels that are more appropriate
to describing why these choices are useful.

\subsection{Definitions and notation}

A {\em kernel} on a set $\mathcal{X}$ is a function $k : \mathcal{X}
\times \mathcal{X} \rightarrow \bbR$.  We will restrict our attention
to {\em symmetric} kernels, for which $k(x,y) = k(y,x)$.  Some kernels
depend on additional {\em hyper-parameters},~e.g.~the length scale and
smoothness parameters described for squared exponential and Mat\'ern
kernels in the previous section.

Often the set $\mathcal{X}$ has special structure, and we want kernels
to have invariants that are natural to that structure.  For
$\mathcal{X} = \bbR^d$, we like to think about invariance under
translation and rotation.  We say $k$ is {\em stationary} if it is
invariant under translation, i.e. $k(x+u,y+u)$ for any $u$.  If
$k(x,y)$ depends only on $x$ and the distance between $x$ and $y$, we
say it is {\em isotropic}.  When $k$ is both stationary and isotropic,
we often identify it with a {\em radial basis function} $\phi$,
i.e. $k(x,y) = \phi(\|x-y\|)$.

For ordered $X, Y \subset \mathcal{X}$, we write the
matrix of pairwise evaluations as
\[
  (K_{XY})_{ij} = k(x_i,y_j).
\]
We say $K_{XX}$ is the {\em kernel matrix} for the set $X$ and kernel
$k$.  The kernel function $k$ is {\em positive definite} if
$K_{XX}$ is positive definite whenever $X$ consists of distinct
points.  The squared exponential and absolute exponential kernels
are positive definite; the cubic spline and thin plate spline
kernels are not.

A kernel is {\em conditionally positive definite} relative
to a space of functions $\mathcal{U}$ from $\mathcal{X} \rightarrow
\mathcal{R}$ if
\[
  v^T K_{XX} v \geq 0
  \mbox{ whenever } v \neq 0
  \mbox{ and } \forall u \in \mathcal{U}, ~ u_X^T v = 0.
\]
A kernel on $\mathcal{X} = \bbR^n$ is {\em conditionally positive
  definite of order $d$} if it is conditionally positive definite
relative to the space $\mathcal{P}_{d-1}$ of polynomials of total
degree at most $d-1$.  The cubic spline and thin plate spline kernels
are both conditionally positive definite of order 2.

\section{Feature maps}

Suppose we want to approximate $f : \bbR^n \rightarrow \bbR$.
A very simple scheme is to approximate $f$ by a {\em linear} function,
\[
  f(x) \approx \hat{f}(x) = c^T x
\]
where the coefficients $c$ are determined from $m \geq n$ samples by
a least squares fitting method.  That is, we solve
\[
  \mbox{minimize } \|X^T c-f_X\|^2
\]
where $X$ is a matrix whose columns are data points $x_1, \ldots, x_m$
and $f_X$ is a vector of function values $f(x_1), \ldots, f(x_m)$.

Unfortunately, the space of linear functions is rather limited, so we
may want a richer class of models.  The next step up in complexity
would be to look at a vector space $\mathcal{H}$ of functions from
$\bbR^n \rightarrow \bbR$, with basis functions
$\psi_1, \ldots, \psi_N$ that we collect into a single
vector-valued function $\psi$.
For example, for one-dimensional function approximation on $[-1,1]$,
we might choose a polynomial space of approximating functions
$\mathcal{H} = \mathcal{P}_{N-1}$ and use the Chebyshev basis functions
$T_0(x), \ldots, T_{N-1}(x)$.
I think of these as basis vectors for a space
of candidate approximating functions, but in the language of machine
learning, we say that the functions $\psi_i$ are {\em features} and
the vector-valued function $\psi$ is a {\em feature map}.  We
write our approximation as
\[
  \hat{f}(x) = c^T \psi(x) = \sum_{j=1}^N c_j \psi_j(x),
\]
and if we have function values at $m \geq N$ points, we can again fit
the coefficients $c$ by the least squares problem
\[
  \mbox{minimize} \|\Psi^T c - f_X\|^2
\]
where $[\Psi]_{ij} = \psi_i(x_j)$.  All nonlinearity in the
scheme comes from the nonlinear functions in $\psi$;
afterward, we have a standard linear problem.

What if the dimension of our approximating space $\mathcal{H}$ is much
larger than the amount of data we have --- or even if it is infinite
dimensional?  Or, equivalently: what if we have more features than
training points?  In this case, many different approximations in the
space all fit the data equally well, and we need a rule to choose from
among them.  From a linear numerical algebra perspective, a natural
choice is the {\em minimal norm} solution; that is, we
approximate $\hat{f}(x) = c^T \psi(x)$ as before, but choose the
coefficients to minimize $\|c\|$ subject to $\Psi^T c = f_X$.
The solution to this linear system is
\[
  c = \Psi (\Psi^T \Psi)^{-1} f_X
\]
and therefore
\[
  \hat{f}(x) = \psi(x)^T \Psi (\Psi^T \Psi)^{-1} f_X.
\]
We now observe that all the entries of the vector $\psi(x)^T \Psi$
and the Gram matrix $\Psi^T \Psi$ can be written in terms of inner
products between feature vectors.  Let $k(x,y) = \psi(x)^T \psi(y)$;
then
\[
  \hat{f}(x) = k_{xX} K_{XX}^{-1} f_X
\]
where $k_{xX}$ denotes the row vector with entries $k(x,x_i)$
and $[K_{XX}]_{ij} = k(x_i,x_j)$.
The function $k(x,y) = \psi(x)^T \psi(y)$ is the kernel function,
and this way of writing the approximation only in terms of
these inner products, without writing down a feature map, is
sometimes called the ``kernel trick.''

So far we have shown how to get from feature maps to kernels; what if
we want to go in the other direction?  As it happens, if we are given
a positive (semi)definite kernel function, we can always construct a
(possibly infinite) feature map associated with the kernel.  To do
this, we define an integral operator on an appropriate space of
distributions $g$
\[
  [\mathcal{K}g](x) = \int k(x,y) g(y) \, dy.
\]
This encodes all the information about the kernel --- formally, for
example, if we let $g$ be a Dirac delta at $x_i$, then
$[\mathcal{K}g](x) = k(x,x_i)$.
{\em Mercer's theorem} tell us there is an eigenvalue
decomposition with eigenpairs $(\lambda_j, v_j(x))$ so that
\[
  [\mathcal{K} g](x) =
  \sum_{j=1}^\infty \lambda_j v_j(x) \left[ \int v_j(y) g(y) \, dy \right],
\]
and the features $\psi_j(x) = \sqrt{\lambda_j} v_j(x)$ give us the kernel.

\section{Kernels as basis functions}

As we have seen in the previous section, a standard idea for function
approximation is to choose a function from some fixed approximation
space $\mathcal{H}$.  For example, if we define the function
\[
  \phi(r) = \exp\left( -\frac{1}{2} \left( \frac{r}{l} \right)^2 \right),
\]
then a reasonable space for functions on $[0,1]$ might consist of
approximants that are combinations of these radial basis function
``bumps'' centered at each of the nodes $j/N$ for $j = 0, \ldots, N$:
\[
  \hat{f}(x) = \sum_{j=0}^N c_j \phi(x-j/N).
\]
If we have data on a set of $N+1$ points $X$, and $X'$ denotes the points
$0, 1/N, \ldots, N$, then the interpolation equations
$\hat{f}_X = f_X$ take the form
\[
  \hat{f}(x_i) = \sum_{j=0}^n \phi(x_i-x'_j) c_j = f(x_i)
\]
which we write in matrix form as
\[
  K_{XX'} c = f_X
\]
where $k(x,y) = \phi(|x-y|)$.  With more data points, we use least
squares.

What if the data points are not uniformly distributed on $[0,1]$?  For
example, what if we have more data points close to 1 than we have close to 0?
The choice of a fixed approximation space limits us: we have no way of
saying that with more data close to 1, we should allow the
approximation more wiggle room to fit the function there.  We can do
this by putting more basis functions that are ``centered'' in the area
where the data points are dense, e.g.~by making the data points and
the centers coincide:
\[
  \hat{f}(x) = \sum_{j=1}^N c_j \phi(|x-x_j|).
\]
Then the interpolation conditions are
\[
   \sum_{j=1}^N \phi(|x_i-x_j|) c_j = f(x_i).
\]
or $K_{XX} c = f_X$.  By adapting the space to allow more flexibility
close to where the data is, we hope to get better approximations, but
there is another advantage as well: if the kernel is positive definite
then the matrix $K_{XX}$ is symmetric and positive definite, and we
need not worry about singularity of the linear system.

Of course, nothing says we are not allowed to use an approximation
space that is adapted to the sample points {\em as well as} a fixed
approximation space.  We saw this already in our discussion of cubic
splines, where we had a piece associated with the cubic radial basis
function together with a linear ``tail'':
\[
  \hat{f}(x) = \sum_{j=1}^m c_j |x-x_j|^3 + d_1 + d_2 x.
\]
In order to uniquely solve the linear system, we need some additional
constraints; for cubic splines, we use the
{\em discrete orthogonality condition}
\[
  \sum_j c_j p(x_j) = 0, \mbox{ any linear } p(x).
\]
More generally, if a kernel is {\em conditionally positive definite}
relative to a space of functions $\mathcal{U}$ spanned by the basis
$p_1(x), \ldots, p_{m'}(x)$, then the coefficients of the
approximation
\[
  \hat{f}(x) = \sum_j c_j k(x,x_j) + \sum_j d_j p_j(x)
\]
are determined by the interpolation conditions and the
discrete orthogonality condition
\[
  \sum_j c_j u(x_j) = 0, \forall u \in \mathcal{U}.
\]
We can write these conditions in matrix form as
\[
  \begin{bmatrix} K_{XX} & P \\ P^T & 0 \end{bmatrix}
  \begin{bmatrix} c \\ d \end{bmatrix} =
  \begin{bmatrix} f_X \\ 0 \end{bmatrix}
\]
where $[P]_{ij} = p_j(x_i)$.  So long as $P$ is full rank (we call
this well-posedness of the points for interpolation in $\mathcal{U}$),
this linear system is invertible.  Most often, we include polynomial
tails to guarantee solvability of the interpolation problem with
conditionally positive definite kernels, but nothing prevents us from
incorporating such terms in other approximations as well --- and,
indeed, it may do our approximations a great deal of good.

\section{Kernels and quadratic forms}

% Positive definiteness and kernel 

\subsection{From feature maps to RKHS}

Let us return again to the feature map picture of kernels: we have an
approximation space $\mathcal{H}$, which we will assume for the moment
is finite dimensional, with a basis $\psi_1(x), \psi_2(x), \ldots,
\psi_N(x)$.  From this space, we seek an approximation of the form
\[
  \hat{f}(x) = \sum_i c_i \psi_i(x)
\]
so that $\sum_i c_i^2$ is minimal subject to the data constraints.
Hidden in this construction is that we have implicitly defined an
{\em inner product} for the space of functions $\mathcal{H}$
for which the $\{ \psi_i \}_{i=1}^N$ basis is orthonormal; that is,
\[
  \langle \psi_i, \psi_j \rangle_{\mathcal{H}} = \delta_{ij},
\]
and we can rephrase the approximation problem without direct reference to
the expansion coefficients as
\[
  \mbox{minimize } \|\hat{f}\|_{\mathcal{H}}^2
  \mbox{ s.t.~} \hat{f}_X = f_X.
\]
We can also express evaluation of $\hat{f}$ at a point in terms of the
inner product; if we define $k_y \in \mathcal{H}$ by
\[
  k_y(x) = \sum_{i=1}^N \psi_i(y) \psi_i(x),
\]
then
\begin{align*}
  \left\langle \hat{f}, k_y \right\rangle_{\mathcal{H}} 
   &= \left\langle \sum_{i=1}^N c_i \psi_i,
             \sum_{j=1}^N \psi_j(y) \psi_j \right\rangle_{\mathcal{H}} \\
   & = \sum_{i,j} c_i \psi_j(y) \left\langle \psi_i, \psi_j \right\rangle_{\mathcal{H}} 
   = \sum_{i=1}^N c_i \psi_i(y) = \hat{f}(y).
\end{align*}
This idea applies more generally: we can write point evaluation at $y$
for any function $g \in \mathcal{F}$ as $\langle g, k_y \rangle_{\mathcal{H}}$,
where the feature vector $\psi(y)$ gives the coefficients for the
evaluation function $k_y$.  Note that
$k(x,y) = \langle k_x, k_y \rangle_{\mathcal{H}} = \psi(x)^T \psi(y)$.

So far, we have only described what happens with finite-dimensional
vector spaces of approximating functions.  But though there are some
technicalities that we must deal with in the infinite-dimensional
case, the finite-dimensional picture sets the stage for the more
general definition.
A space of functions $\mathcal{H}$ is a
{\em reproducing kernel Hilbert space} (RKHS)
if function evaluation can be written in terms of the inner product:
\[
  g(x) = \langle g, k_x \rangle_{\mathcal{H}}
  \mbox{ for any } g \in \mathcal{H},
\]
where $k_x(y) = k(x,y) = k(y,x) = k_y(x)$ is the associated
(reproducing) kernel, so called because it reproduces function
evaluations.  Given an orthonormal basis (a feature map)
for $\mathcal{H}$, we can write the inner product on $\mathcal{H}$
and the inner product.  Of course, we often would prefer not to use
feature maps for concrete computation; is there another way to
get our hands on the inner product?  The answer, as it turns out,
is yes.

\subsection{From kernels to inner products}

Every RKHS has an associated kernel function.  We now sketch the path
going in other direction: given a positive definite kernel function,
reconstruct the RKHS.  The key observation is that for any finite set
of points $X$,
\[
  \left\langle
    \sum_i c_i k_{x_i}, \sum_j d_j k_{x_j}
  \right\rangle_{\mathcal{H}}
  = \sum_{i,j} c_i d_j k(x_i,x_j)
  = c^T K_{XX} d.
\]
This allows us to write the inner products between any functions that
can be expressed as a linear combination of kernel shapes centered at
any {\em finite} number of points.  This is a rich set of functions,
almost rich enough to get a full RKHS.  It is not quite enough:
technically, the set of functions that can be written in terms of a
finite number of centers is a {\em pre-Hilbert space}; to get the rest
of the way to a RKHS, we need to ``complete'' the space by including
limits of Cauchy sequences.  The RKHS constructed in this way
is sometimes called the {\em native space} for the associated kernel
or radial basis function.
Unfortunately, it is not always easy to characterize the functions
in the native space for a kernel!  The native space for the squared
exponential kernel is small, consisting of only very smooth functions,
while the native spaces of some other kernels are larger.

\subsection{Error analysis of kernel interpolation}

Let us approximate $f \in \mathcal{H}$ by kernel interpolation
at $X = (x_1, \ldots, x_n)$, i.e.
\[
  \mbox{minimize } \|\hat{f}\|_{\mathcal{H}}^2
  \mbox{ s.t.~} \hat{f}_X = f_X.
\]
What is the error $|f(y)-\hat{f}(y)|$ at a new point $y$?
Our approach looks like the approach to error analysis for polynomial
interpolation:
\begin{enumerate}
\item Define a function $\tilde{f}$ by interpolating at one more point.
\item Use regularity to control the difference between $\tilde{f}$ and $\hat{f}$.
\end{enumerate}
Specifically, let $\tilde{f}$ interpolate at $X' = (x_1, \ldots, x_n, y)$:
\[
  \mbox{minimize } \|\tilde{f}\|_{\mathcal{H}}^2
  \mbox{ s.t.~} \tilde{f}_X = f_X \mbox{ and } \tilde{f}(y) = f(y).
\]
Because $\hat{f}$ and $\tilde{f}$ involve the same optimization
objective, but with more constraints for $\tilde{f}$ (and $f$ involves
a limiting case of even more constraints), we have
\[
  \|\hat{f}\|_{\mathcal{H}}^2 \leq
  \|\tilde{f}\|_{\mathcal{H}}^2 \leq
  \|f\|_{\mathcal{H}}^2.
\]
The advantage of this is that we can write the norms in this
inequality in a nice way.  Observe that $e = \tilde{f}-\hat{f}$ is a
kernel interpolant with centers at $X'$ and coefficients
$d = K_{X'X'}^{-1} e_{X'}$; 
\[
  \|e\|_{\mathcal{H}}^2
    = d^T K_{X'X'} d = e_{X'}^T K_{X'X'}^{-1} e_{X'}
\]
Now we use the fact that $e_{X'}$ is zero except in the last component
$e(y)$, and we know how to write the last diagonal element of the
inverse of a matrix:
\[
  \|e\|_{\mathcal{H}}^2 = [K_{X'X'}^{-1}]_{yy} e(y)^2
\]
Therefore, we have
\[
  |e(y)| = P_X(y) \|e\|_{\mathcal{H}}, \quad
  \mbox{where } P_X(y)^2 = k_{yy} - k_{yX} K_{XX}^{-1} k_{Xy}.
\]
The function $P$ is known as the {\em power function} in some
communities.  Now, we observe that
$\langle e, \hat{f} \rangle_{\mathcal{H}} = 0$ by construction,
so the Pythagorean theorem gives us
\[
  \|\hat{f}\|_{\mathcal{H}}^2 + \|e\|_{\mathcal{H}}^2 =
  \|\tilde{f}\|_{\mathcal{H}}^2.
\]
Combining with the bound $\|\tilde{f}\|_{\mathcal{H}} \leq \|f\|_{\mathcal{H}}$,
we have
\[
  \|e\|_{\mathcal{H}}^2 \leq \|f\|_{\mathcal{H}}^2 - \|\hat{f}\|_{\mathcal{H}}^2.
\]
Putting all the pieces together, we have the error bound
\[
  |f(y)-\hat{f}(y)| = |e(y)| \leq
    P_X(y) \sqrt{\|f\|_{\mathcal{H}}^2 - \|\hat{f}\|_{\mathcal{H}}^2}.
\]

\subsection{Beyond the positive definite}

So far in this section, we have only considered positive definite
kernels.  What about {\em conditionally} positive definite kernels,
like the thin plate spline or cubic spline kernels?  For a kernel that
is conditionally positive definite with respect to a space
$\mathcal{U}$, we define a pre-Hilbert space of interpolants with
a tail in $\mathcal{U}$, i.e.
\[
  \check{\mathcal{H}} =
  \left\{ g(x) = \sum_{j=1}^{|X|} c_j k(x,x_j) + u(x) :
    c \in \bbR^{|X|}, u \in \mathcal{U},
    \mbox{ and } \forall v \in \mathcal{U}, c_X^T v_X = 0 \right\}.
\]
We define the quadratic form as in the positive definite case
(e.g.~$c^T K_{XX} c$), but now the quadratic form is only
{\em semi-definite}, as it will be zero for functions that
lie in $\mathcal{U}$.  We can complete $\check{\mathcal{H}}$
under this {\em semi-norm}, and the remainder of the analysis
goes forward as with the positive definite case, except with some more
technicalities.

There is a reason to think about the conditionally positive definite
case, though: the cubic spline is conditionally positive definite,
and for the cubic spline we can give a mechanical intuition behind the
rather formal-looking error analysis in the previous subsection.
A cubic spline corresponds to the interpolant we would get if we bent a thin
beam of wood to the shape of our data.  This shape minimizes the
{\em bending energy}
\[
  \mathcal{E}[u] = |u|_{\mathcal{H}}^2 = \frac{1}{2} \int |u''(x)|^2 \, dx.
\]
When we interpolate $f$ at points $X$, we use a certain amount of
energy to bend the beam to the right shape; this energy is less than
the total bending energy of $f$.  If we want to push the beam at $y$
from its minimal-energy position $\hat{f}(y)$ to a new position
$\hat{f}(y) + e(y)$, we use an amount of energy associated with the
stiffness times $|e(y)|^2$.  This additional energy or work to bend
$\hat{f}$ to become $\tilde{f}$ can be no more than the difference in
the bending energy of $\hat{f}$ and the bending energy of $f$.

\section{Gaussian processes}

Our final story comes from {\em Gaussian processes} (GP).  Informally, just
as the ordinary Gaussian distribution is over numbers and multivariate
Gaussian distributions are over vectors, a Gaussian process is a
distribution over functions.  More formally, a Gaussian process on a
set $\mathcal{X}$ with mean field $\mu$ and covariance kernel $k$ is a
collection of Gaussian random variables indexed by $\mathcal{X}$, any
finite subset of which obeys a multi-variate Gaussian distribution;
that is, if $f$ is a draw from a Gaussian process, then for any
$X \subset \mathcal{X}$,
\[
  f_X \sim N(\mu_X, K_{XX}).
\]
In Bayesian inference using GPs, we start with a {\em prior} GP from
which we assume $f$ is drawn, then compute a {\em posterior} GP
conditioned on data.

Because GPs are defined in terms of the behavior at finite subsets of
points, we can really focus on the multivariate normal case.
Suppose a multivariate Gaussian random variable $Y$ is
partitioned into $Y_1$ and $Y_2$.  For simplicity, we assume
$Y$ has mean zero.  Then the distribution is
\[
  Y \sim N(0, K),
\]
i.e.~we have the probability density
\[
p(y) = \frac{1}{\det(2\pi K)}
  \exp\left( -\frac{1}{2} y^T K^{-1} y \right).
\]
Now, we rewrite the quadratic form $y^T \Omega y$ (where
$\Omega = K^{-1}$ is the {\em precision matrix})
in terms of the components $y_1$ and $y_2$:
\[
  y^T K^{-1} y
  = y^T \Omega y
  = (y_2 - z) \Omega_{22} (y_2-z), \quad \mbox{ where }
  z = -\Omega_{22}^{-1} \Omega_{21} y_2  
\]
Now we rewrite the $(2,1)$ block of the equation $\Omega K = I$
as $\Omega_{21} K_{11} + \Omega_{22} K_{21} = 0$, then
rearrange to $-\Omega_{22}^{-1} \Omega_{21} = K_{21} K_{11}^{-1}$,
to get the more convenient formula
\[
  z = K_{21} K_{11}^{-1} y_1.
\]
The same approach gives us the Schur complement relation
$\Omega_{22}^{-1} = K_{22}-K_{21} K_{11}^{-1} K_{12} = S$.
Plugging this formulation of the quadratic into the joint density and
dividing out by the marginal for $y_1$ gives the conditional density
\[
p(y_2 | y_1)
  = \frac{p(y_1, y_2)}{\int p(y_1, w) \, dw}
  = \frac{1}{\sqrt{\det(2\pi S)}} \exp\left( -\frac{1}{2} (y_2-z)^T S^{-1} (y_2-z) \right).
\]
Thus, the conditional distribution for $Y_2$ given $Y_1 = y_1$ is
again Gaussian:
\[
  Y_2 | Y_1 = y_1 \quad \sim \quad
  N(z, S).
  \]
  
Applying the same logic to Gaussian processes, we find that if
$f$ is drawn from a GP with mean field $0$ and covariance kernel $k$,
then conditioning on observations at points $X$ gives a new GP
with mean and covariance kernel
\[
  \hat{\mu}(x) = k_{xX} K_{XX}^{-1} f_X
  \quad \mbox{and} \quad
  \hat{k}(x,x') = k(x,x') - k_{xX} K_{XX}^{-1} k_{Xx'}.
\]
The conditional mean field $\hat{\mu}(x)$ is exactly the same as
the kernel prediction that we have seen derived in other ways,
and we might recognize the {\em preditive variance at $x$}
conditioned on data at $X$ as
\[
  \hat{k}(x,x) = k(x,x) - k_{xX} K_{XX}^{-1} k_{Xx} = P_X(x)^2,
\]
where $P_X(x)$ is the power function for $x$ that we saw
in the last section.

\section{Deterministic or stochastic?}

In the previous two sections, we have developed two ways to think
about the error analysis of kernel methods to interpolate $f$:
\begin{enumerate}
\item {\em Optimal approximation}: suppose $\|f\|_{\mathcal{H}} \leq C$ and let
  \begin{align*}
  \mathcal{F} &=
    \left\{
    g \in \mathcal{H} : \|g\|_{\mathcal{H}} \leq C \mbox{ and } g_X = f_X
    \right\} \\
    &=
    \left\{
      \hat{f} + u \in \mathcal{H} :
      \langle \hat{f}, u \rangle_{\mathcal{H}} = 0 \mbox{ and }
      \|u\|_{\mathcal{H}}^2 \leq C^2 - \|\hat{f}\|_{\mathcal{H}}^2
    \right\}.
  \end{align*}
  That is, $\hat{f}$ is the center point of a region that is both
  consistent with the data constraints and the norm constraints.
  Because it is at the center of this set, $\hat{f}$
  minimizes the {\em worst possible error} (in the native space norm)
  over all possible $f$
  that are consistent with what we know.  To get pointwise error
  estimates, we look at bounds on $|u(x)|^2$ for all possible $u$ that
  satisfy $u_X = 0$ and $\|u\|_{\mathcal{H}}^2 \leq C^2 -
  \|\hat{f}\|_{\mathcal{H}}^2$.
\item {\em Bayesian inference}: suppose $f$ is drawn from a Gaussian
  process with some known covariance kernel.  Conditioned on the data
  $f_X$, we have a new Gaussian process with mean field $\hat{f}$
  and a conditional kernel.  To get pointwise error estimates, we look
  at the predictive distribution at a new test point $x$ (itself a
  Gaussian distribution), including the predictive variance.
\end{enumerate}
The deterministic approach gives us the {\em best worst-case error}
given what we know about the function; the Bayesian approach gives us
an {\em expected value}.  Both give the same predictions, and both
use the same quantities in computing an error result, though with
different interpretations.  Both approaches also use information that might
not be easy to access (a bound on a native space norm of $f$, or an
appropriate prior distribution).

Why should we not just pick one approach or the other?  Apart from the
question of modeling assumption, the two approaches yield different
predictions if the measurements of $f$ are more complex, e.g.~if we
have information such as an inequality bound or a nonlinear
relationship between point values of $f$.  However, this is beyond the
scope of the current discussion.

\end{document}
