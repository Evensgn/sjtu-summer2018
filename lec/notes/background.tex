\documentclass[12pt, leqno]{article}
\usepackage{fancyhdr}
\usepackage[sort&compress]{natbib}
\usepackage[letterpaper=true,colorlinks=true,linkcolor=black]{hyperref}

\usepackage{amsfonts}
\usepackage{amsmath}
\usepackage{amssymb}
\usepackage{color}
\usepackage{tikz}
\usepackage{pgfplots}
\usepackage{listings}
%\usepackage{courier}
%\usepackage[utf8]{inputenc}
%\usepackage[russian]{babel}

\lstset{
  numbers=left,
  basicstyle=\ttfamily\footnotesize,
  numberstyle=\tiny\color{gray},
  stepnumber=1,
  numbersep=10pt,
}

\newcommand{\iu}{\ensuremath{\mathrm{i}}}
\newcommand{\bbR}{\mathbb{R}}
\newcommand{\bbC}{\mathbb{C}}
\newcommand{\calV}{\mathcal{V}}
\newcommand{\calW}{\mathcal{W}}
\newcommand{\macheps}{\epsilon_{\mathrm{mach}}}
\newcommand{\matlab}{\textsc{Matlab}}

\newcommand{\ddiag}{\operatorname{diag}}
\newcommand{\fl}{\operatorname{fl}}
\newcommand{\nnz}{\operatorname{nnz}}
\newcommand{\tr}{\operatorname{tr}}
\renewcommand{\vec}{\operatorname{vec}}

\newcommand{\vertiii}[1]{{\left\vert\kern-0.25ex\left\vert\kern-0.25ex\left\vert #1
    \right\vert\kern-0.25ex\right\vert\kern-0.25ex\right\vert}}
\newcommand{\ip}[2]{\langle #1, #2 \rangle}
\newcommand{\ipx}[2]{\left\langle #1, #2 \right\rangle}
\newcommand{\order}[1]{O( #1 )}

\newcommand{\kron}{\otimes}


\newcommand{\hdr}[1]{
  \pagestyle{fancy}
  \lhead{Bindel, Summer 2018}
  \rhead{Numerics for Data Science}
  \fancyfoot{}
  \begin{center}
    {\large{\bf #1}}
  \end{center}
  \lstset{language=matlab,columns=flexible}  
}


\newcommand{\calK}{\mathcal{K}}
\newcommand{\calP}{\mathcal{P}}
\newcommand{\calR}{\mathcal{R}}

\begin{document}
\lstset{language=matlab,columns=flexible}
\hdr{Background Plus a Bit}

For this class, I assume you know some linear algebra and
multivariable calculus.  You should also know how to write and debug
simple MATLAB scripts, or know enough programming to pick it up.
But there are some things you may never have forgotten that you
will need for this class, and there are other things that you might
not have learned.  This set of notes will describe some of these things.
It is fine if you do {\em not} know all this material!  Ask questions
if you see things you do not know or understand, and do not feel bad
about asking the same question more than once if you get confused or
forget during class.

\section{Linear algebra background}

In what follows, I will mostly consider real vector spaces.

\paragraph{Vectors}
You should know a vector as:
\begin{itemize}
\item An object that can be scaled or added to other vectors.
\item A column of numbers, often stored sequentially
  in computer memory.
\end{itemize}
We map between the abstract and concrete pictures of vector spaces
using a basis.  For example, a basis for the vector space of quadratic
polynomials in one variable is $\{1, x, x^2\}$; using this basis, we
might concretely represent a polynomial $1 + x^2/2$ in computer memory
using the coefficient vector
\[
  c = \begin{bmatrix} 1 \\ 0 \\ 0.5 \end{bmatrix}.
\]
In numerical linear algebra, we use column vectors more often than
row vectors, but both are important.  A row vector defines a linear
function over column vectors of the same length.  For example,
in our polynomial example, suppose we want the row vector
corresponding to evaluation at $-1$.  With respect to the power basis
$\{1, x, x^2\}$ for the polynomial space, that would give us the
row vector
\[
  w^T = \begin{bmatrix} 1 & -1 & 1 \end{bmatrix}
\]
Note that if $p(x) = 1+x^2/2$, then
\[
  p(-1) = 1 + (-1)^2/2
  = w^T c
  = \begin{bmatrix} 1 & -1 & 1 \end{bmatrix}
    \begin{bmatrix} 1 \\ 0 \\ 0.5 \end{bmatrix}.
\]

\paragraph{Vector norms and inner products}
A {\em norm} $\|\cdot\|$ measures vector lengths.  It is
positive definite, homogeneous, and sub-additive:
\begin{align*}
  \|v\| & \geq 0 \mbox{ and } \|v\| = 0 \mbox{ iff } v = 0 \\
  \|\alpha v\| &= |\alpha| \|v\| \\
  \|u+v\| & \leq \|u\| + \|v\|.
\end{align*}
The three most common vector norms we work with are the
Euclidean norm (aka the 2-norm), the $\infty$-norm (or max norm),
and the $1$-norm:
\begin{align*}
  \|v\|_2 &= \sqrt{\sum_j |v_j|^2} \\
  \|v\|_\infty &= \max_j |v_j| \\
  \|v\|_1 &= \sum_j |v_j|
\end{align*}
Many other norms can be related to one of these three norms.

An {\em inner product} $\langle \cdot, \cdot \rangle$
is a function from two vectors into the real
numbers (or complex numbers for an complex vector space).  It is
positive definite, linear in the first slot, and symmetric (or
Hermitian in the case of complex vectors); that is:
\begin{align*}
  \langle v, v \rangle & \geq 0 \mbox{ and }
  \langle v, v \rangle = 0 \mbox{ iff } v = 0 \\
%
  \langle \alpha u, w \rangle &= \alpha \langle u, w \rangle
  \mbox{ and }
  \langle u+v, w \rangle = \langle u, w \rangle + \langle v, w \rangle \\
%
  \langle u, v \rangle &= \overline{\langle v, u \rangle},
\end{align*}
where the overbar in the latter case corresponds to complex
conjugation.  Every inner product defines a corresponding norm
\[
  \|v\| = \sqrt{ \langle v, v \rangle}
\]
The inner product and the associated norm satisfy
the {\em Cauchy-Schwarz} inequality
\[
  \langle u, v \rangle \leq \|u\| \|v\|.
\]
The {\em standard inner product} on $\bbR^n$ is
\[
  x \cdot y = y^T x = \sum_{j=1}^n y_j x_j.
\]
But the standard inner product is not the only inner product,
just as the standard Euclidean norm is not the only norm.

\paragraph{Matrices}
You should know a matrix as:
\begin{itemize}
\item A representation of a linear map
\item An array of numbers, often stored sequentially in memory.
\end{itemize}

A matrix can {\em also} represent a bilinear function mapping
two vectors into the real numbers (or complex numbers for complex
vector spaces):
\[
  (v,w) \mapsto w^T A v.
\]
Symmetric matrices also represent {\em quadratic forms}
mapping vectors to real numbers
\[
  \phi(v) = v^T A v
\]
We say a symmetric matrix $A$ is {\em positive definite} if
the corresponding quadratic form is positive definite, i.e.
\[
  v^T A v \geq 0 \mbox{ with equality iff } v = 0.
\]

Many ``rookie mistakes'' in linear algebra involve forgetting
ways in which matrices differ from scalars:
\begin{itemize}
\item
  Not all matrices are square.
\item
  Not all matrices are invertible (even nonzero matrices can be
  singular).
\item
  Matrix multiplication is associative, but not commutative.
\end{itemize}
Don't forget these facts!

\paragraph{Block matrices}
We often partition matrices into submatrices of different
sizes.  For example, we might write
\[
  \begin{bmatrix}
    a_{11} & a_{12} & b_1 \\
    a_{21} & a_{22} & b_2 \\
    c_1 & c_2 & d
  \end{bmatrix} =
  \begin{bmatrix}
    A & b \\
    c^T & d
  \end{bmatrix}, \mbox{ where }
  A = \begin{bmatrix} a_{11} & a_{12} \\ a_{21} & a_{22} \end{bmatrix},
  b = \begin{bmatrix} b_1 \\ b_2 \end{bmatrix},
  c = \begin{bmatrix} c_1 \\ c_2 \end{bmatrix}.
\]
We can manipulate block matrices in much the same way we manipulate
ordinary matrices; we just need to remember that matrix multiplication
does not commute.

\paragraph{Matrix norms}
The matrices of a given size form a vector space, and we can define
a norm for such a vector space the same way we would for any other
vector space.  Usually, though, we want matrix norms that are compatible with
vector space norms (a ``submultiplicative norm''), i.e. something that
guarantees
\[
  \|Av\| \leq \|A\| \|v\|
\]
The most common choice is to use an {\em operator norm}:
\[
  \|A\| \equiv \sup_{\|v\| = 1} \|Av\|.
\]
The operator 1-norm and $\infty$ norm are easy to compute
\begin{align*}
  \|A\|_1 &= \max_j \sum_i |a_{ij}| \\
  \|A\|_\infty &= \max_i \sum_j |a_{ij}|
\end{align*}
The operator 2-norm is theoretically useful, but not so easily computed.

In addition to the operator norms, the {\em Frobenius norm} is a
common matrix norm choice:
\[
  \|A\|_F = \sqrt{ \sum_{i,j} |a_{ij}|^2}
\]

\paragraph{Matrix structure}
We considered many types of {\em structure} for matrices this
semester.  Some of these structures are what I think of as ``linear
algebra structures,'' such as symmetry, skew symmetry, orthogonality,
or low rank.  These are properties that reflect behaviors of an
operator or quadratic form that don't depend on the specific basis for
the vector space (or spaces) involved.  On the other hand, matrices
with special nonzero structure -- triangular, diagonal, banded,
Hessenberg, or sparse -- tend to lose those properties under any but a
very special change of basis.  But these nonzero structures or matrix
``shapes'' are very important computationally.

\paragraph{Matrix products}
Consider the matrix-vector product
\[
  y = Ax
\]
You probably first learned to compute this matrix product with
\[
  y_{i} = \sum_j a_{ij} x_j.
\]
But there are different ways to organize the sum depending on how
we want to think of the product.  We could say that $y_i$ is the
product of row $i$ of $A$ (written $A_{i,:}$) with $x$; or we could
say that $y$ is a linear combination of the columns of $A$,
with coefficients given by the elements of $x$.  Similarly,
consider the matrix product
\[
  C = A B.
\]
You probably first learned to compute this matrix product with
\[
  c_{ij} = \sum_k a_{ik} b_{kj}.
\]
But we can group and re-order each of these sums in different ways,
each of which gives us a different way of thinking about matrix
products:
\begin{align*}
  C_{ij} &= A_{i,:} B_{:,j} & \mbox{(inner product)} \\
  C_{i,:} &= A_{i,:} B & \mbox{(row-by-row)} \\
  C_{:,j} &= A B_{:,j} & \mbox{(column-by-column)} \\
  C &= \sum_k A_{:,k} B_{k,:} & \mbox{(outer product)}
\end{align*}
One can also think of organizing matrix multiplication around a
partitioning of the matrices into sub-blocks.  Indeed, this is how
tuned matrix multiplication libraries are organized.

\paragraph{Fast matrix products}
There are some types of matrices for which we can compute
matrix-vector products very quickly.  For example, if
$D$ is a diagonal matrix, then we can compute $Dx$ with one
multiply operation per element of $x$.  Similarly, if
$A = uv^T$ is a rank-one matrix, we can compute $Ax$ quickly
by recognizing that matrix multiplication is associative
\[
  Ax = (uv^T) x = u (v^T x).
\]
Thus, we can apply $A$ with one dot product (between $v$ and $x$)
and a scaling operation.

\paragraph{Singular values and eigenvalues}
A square matrix $A$ has an eigenvalue $\lambda$ and corresponding eigenvector
$v \neq 0$ if
\[
  Av = \lambda v.
\]
A matrix is {\em diagonalizable} if it has a complete basis of
eigenvectors $v_1, \ldots, v_n$; in this case, we write the
{\em eigendecomposition}
\[
  AV = V\Lambda
\]
where $V = \begin{bmatrix} v_1 & \ldots & v_n \end{bmatrix}$
and $\Lambda = \operatorname{diag}(\lambda_1, \lambda_2, \ldots, \lambda_n)$.
If a matrix is not diagonalizable, we cannot write the
eigendecomposition in this form (we need Jordan blocks and generalized
eigenvectors).  In general, even if the matrix $A$ is real and
diagonalizable, we may need to consider complex eigenvalues and eigenvectors.

A real {\em symmetric} matrix is always diagonalizable with real
eigenvalues, and has an orthonormal basis of eigenvectors $q_1,
\ldots, q_n$, so that we can write the eigendecomposition
\[
  A = Q \Lambda Q^T.
\]
For a nonsymmetric (and possibly rectangular) matrix, the natural
decomposition is often not the eigendecomposition, but
the {\em singular value decomposition}
\[
  A = U \Sigma V^T
\]
where $U$ and $V$ have orthonormal columns (the left and right {\em
  singular vectors}) and $\Sigma = \operatorname{diag}(\sigma_1,
\sigma_2, \ldots)$ is the matrix of {\em singular values}.
The singular values are non-negative; by convention, they should
be in ascending order.

\section{Calculus background}

\paragraph{Taylor approximation in 1D}
If $f : \bbR \rightarrow \bbR$ has $k$ continuous derivatives, then
Taylor's theorem with remainder is
\[
  f(x+z) = f(x) + f'(x) z + \ldots + \frac{1}{(k-1)!} f^{(k-1)}(x) +
           \frac{1}{k!} f^{(k)}(x+\xi)
\]
where $\xi \in [x, x+z]$.  We usually work with simple linear
approximations, i.e.
\[
  f(x+z) = f(x) + f'(x) z + O(z^2),
\]
though sometimes we will work with the quadratic approximation
\[
  f(x+z) = f(x) + f'(x) z + \frac{1}{2} f''(x) z^2 + O(z^3).
\]
In both of these, when say the error term $e(z)$ is $O(g(z))$, we mean
that for small enough $z$, there is some constant $C$ such that
\[
  |e(z)| \leq C g(z).
\]
We don't need to remember a library of Taylor expansions, but it is
useful to remember that for $|\alpha| < 1$, we have the geometric series
\[
  \sum_{j=0}^\infty \alpha^j = (1-\alpha)^{-1}.
\]

\paragraph{Taylor expansion in multiple dimensions}
In more than one space dimension, the basic picture of Taylor's
theorem remains the same.  If $f : \bbR^n \rightarrow \bbR^m$, then
\[
  f(x+z) = f(x) + f'(x) z + O(\|z\|^2)
\]
where $f'(x) \in \bbR^{m \times n}$ is the {\em Jacobian matrix}
at $x$.  If $\phi : \bbR^n \rightarrow \bbR$, then
\[
  \phi(x+z) = \phi(z) + \phi'(x) z + \frac{1}{2} z^T \phi''(z) z + O(\|z\|^3).
\]
The row vector $\phi'(x) \in \bbR^{1 \times n}$ is the derivative of
$\phi$, but we often work with the {\em gradient} $\nabla \phi(x) =
\phi'(x)^T$.  The {\em Hessian} matrix $\phi''(z)$ is the matrix of
second partial derivatives of $\phi$.  Going beyond second order
expansion of $\phi$ (or going beyond a first order expansion of $f$)
requires that we go beyond matrices and vectors to
work with tensors involving more than two indices.  For this class,
we're not going there.

\paragraph{Variational notation}
A {\em directional derivative} of a function $f : \bbR^n \rightarrow
\bbR^m$ in the direction $u$ is
\[
  \frac{\partial f}{\partial u}(x) \equiv
  \left. \frac{d}{ds} \right|_{s=0} f(x+su) = f'(x) u.
\]
A nice notational convention, sometimes called {\em variational}
notation (as in ``calculus of variations'') is to write
\[
  \delta f = f'(x) \delta u,
\]
where $\delta$ should be interpreted as ``first order change in.''
In introductory calculus classes, this sometimes is called
a total derivative or total differential, though there one usually
uses $d$ rather than $\delta$.  There is a good reason for using
$\delta$ in the calculus of variations, though, so that's typically
what I do.

Variational notation can tremendously simplify the calculus
book-keeping for dealing with multivariate functions.  For example,
consider the problem of differentiating $A^{-1}$ with respect to
every element of $A$.  I would compute this by thinking of the
relation between a first-order change to $A^{-1}$ (written
$\delta [A^{-1}]$) and a corresponding first-order change to $A$
(written $\delta A$).  Using the product rule and differentiating
the relation $I = A^{-1} A$, we have
\[
  0 = \delta [A^{-1} A] = \delta [A^{-1}] A + A^{-1} \delta A.
\]
Rearranging a bit gives
\[
  \delta [A^{-1}] = -A^{-1} [\delta A] A^{-1}.
\]
One {\em can} do this computation element by element, but it's harder
to do it without the computation becoming horrible.

\paragraph{Matrix calculus rules}
There are some basic calculus rules for expressions involving matrices
and vectors that are easiest to just remember.  These are naturally
analogous to the rules in 1D.  If $f$ and $g$ are differentiable maps
whose composition makes sense, the multivariate chain rule says
\[
  \delta [f(g(x))] = f'(g(x)) \delta g, \quad
  \delta g = g'(x) \delta x
\]
If $A$ and $B$ are matrix-valued functions, we also have
\begin{align*}
  \delta [A+B] &= \delta A + \delta B \\
  \delta [AB] &= [\delta A] B + A [\delta B], \\
  \delta [A^{-1} B] &= -A^{-1} [\delta A] A^{-1} B + A^{-1} \delta B
\end{align*}
and so forth.  The big picture is that the rules of calculus work as
well for matrix-valued functions as for scalar-valued functions,
and the main changes account for the fact that matrix multiplication
does not commute.  You should be able to convince yourself of the
correctness of any of these rules using the component-by-component
reasoning that you most likely learned in an introductory calculus
class, but using variational notation (and the ideas of linear
algebra) simplifies life immensely.

A few other derivatives are worth having at your fingertips
(in each of the following formulas, $x$ is assumed variable
while $A$ and $b$ are constant
\begin{align*}
  \delta [Ax-b] &= A \delta x \\
  \delta [\|x\|^2] &= 2 x^T \delta x \\
  \delta \left[\frac{1}{2} x^T A x - x^T b\right] &= (\delta x)^T (Ax-b) \\
  \delta \left[\frac{1}{2} \|Ax-b\|^2 \right] &= (A \delta x)^T (Ax-b)
\end{align*}
and if $f : \bbR^n \rightarrow \bbR^n$ is given by $f_i(x) = \phi(x_i)$,
then
\[
  \delta [f(x)] = \operatorname{diag}(\phi'(x_1), \ldots, \phi'(x_n))
  \, \delta x.
\]


\section{CS background}

\paragraph{Order notation and performance}
Just as we use big-O notation in calculus to denote a function
(usually an error term) that goes to zero at a controlled rate as the
argument goes to zero, we use big-O notation in algorithm analysis to
denote a function (usually run time or memory usage) that grows at a
controlled rate as the argument goes to infinity.  For instance,
if we say that computing the dot product of two length $n$ vectors
is an $O(n)$ operation, we mean that the time to compute the dot
products of length greater than some fixed constant $n_0$ is bounded
by $C n$ for some constant $C$.  The point of this sort of analysis
is to understand how various algorithms scale with problem size
without worrying about all the details of implementation and
architecture (which essentially affect the constant $C$).

Most of the major factorizations of {\em dense} numerical linear
algebra take $O(n^3)$ time when applied to square $n \times n$
matrices, though some building blocks (like multiplying a matrix
by a vector or scaling a vector) take $O(n^2)$ or $O(n)$ time.
We often write the algorithms for factorizations that take $O(n^3)$
time using block matrix notation so that we can build these
factorizations from a few well-tuned $O(n^3)$ building blocks,
the most important of which is matrix-matrix multiplication.

\paragraph{Graph theory and sparse matrices}
In {\em sparse} linear algebra, we consider matrices that can be
represented by fewer than $O(n^2)$ parameters.  That might mean most
of the elements are zero (e.g.~as in a diagonal matrix), or it might
mean that there is some other low-complexity way of representing the
matrix (e.g.~the matrix might be a rank-1 matrix that can be
represented as an outer product of two length $n$ vectors).  We
usually reserve the word ``sparse'' to mean matrices with few
nonzeros, but it is important to recognize that there are other
{\em data-sparse} matrices in the world.

The {\em graph} of a sparse matrix $A \in \bbR^{N \times N}$ consists
of a set of $N$ vertices $\mathcal{V} = \{1, 2, \ldots, N\}$ and a set
of edges $\mathcal{E} = \{(i,j) : a_{ij} \neq 0\}$.  While the cost of
general dense matrix operations usually depends only on the sizes of
the matrix involved, the cost of sparse matrix operations can be
highly dependent on the structure of the associated graph.

\section{MATLAB background}

\paragraph{Building matrices and vectors}
MATLAB gives us several standard matrix and vector construction functions.
\begin{lstlisting}
  I = eye(n);    % Build n-by-n identity
  Z = zeros(n);  % n-by-n matrix of zeros
  b = rand(n,1); % n-by-1 random matrix (uniform)
  e = ones(n,1); % n-by-1 matrix of ones
  D = diag(e);   % Construct a diagonal matrix
  e2 = diag(D);  % Extract matrix diagonal
\end{lstlisting}

\paragraph{Concatenating matrices and vectors}
In addition to functions for constructing specific types of matrices
and vectors, MATLAB lets us put together matrices and vectors by
horizontal and vertical concatenation.  This works with
matrices just as well as with vectors!
\begin{lstlisting}
  x = [1; 2];       % Column vector
  y = [1, 2];       % Row vector
  M = [1, 2; 3, 4]; % 2-by-2 matrix
  M = [I, A];       % Horizontal matrix concatenation
\end{lstlisting}

\paragraph{Transpose and rearrangemenent}
MATLAB lets us rearrange the data inside a matrix or vector in
a variety of ways.  In addition to the usual transposition
operation, we can also do ``reshape'' operations that let us
interpret the same data layout in computer memory in different ways.
\begin{lstlisting}
  % Reshape A to a vector, then back to a matrix
  % Note: MATLAB is column-major
  avec = reshape(A, prod(size(A)));
  A = reshape(avec, n, n);
  
  A = A';  % Conjugate transpose
  A = A.'; % Simple transpose

  idx = randperm(n);  % Random permutation of indices
  Ac = A(:,idx);      % Permute columns of A
  Ar = A(idx,:);      % Permute rows of A
  Ap = A(idx,idx);    % Permute rows and columns
\end{lstlisting}

\paragraph{Submatrices, diagonals, and triangles}
MATLAB lets us extract specific parts of a matrix, like the diagonal
entries or the upper or lower triangle.
\begin{lstlisting}
  A = randn(6,6);   % 6-by-6 random matrix
  A(1:3,1:3)        % Leading 3-by-3 submatrix
  A(1:2:end,:)      % Rows 1, 3, 5
  A(:,3:end)        % Columns 3-6
  
  Ad = diag(A);       % Diagonal of A (as vector)
  A1 = diag(A,1);     % First superdiagonal
  Au = triu(A);       % Upper triangle
  Al = tril(A);       % Lower triangle
\end{lstlisting}

\paragraph{Matrix and vector operations}
MATLAB provides a variety of {\em elementwise} operations as well as
linear algebraic operations.  To distinguish elementwise
multiplication or division from matrix multiplication and linear
solves or least squares, we put a dot in front of the elementwise
operations.
\begin{lstlisting}
  y = d.*x;   % Elementwise multiplication of vectors/matrices
  y = x./d;   % Elementwise division
  z = x + y;  % Add vectors/matrices
  z = x + 1;  % Add scalar to every element of a vector/matrix
  
  y = A*x;    % Matrix times vector
  y = x'*A;   % Vector times matrix
  C = A*B;    % Matrix times matrix

  % Don't use inv!
  x = A\b;    % Solve Ax = b *or* least squares
  y = b/A;    % Solve yA = b or least squares
\end{lstlisting}

\paragraph{Things best avoided}
There are few good reasons to compute explicit matrix inverses or
determinants in numerical computations.  MATLAB does provide these
operations.  But if you find yourself typing {\tt inv} or {\tt det} in
MATLAB, think long and hard.  Is there an alternate formulation?
Could you use the forward slash or backslash operations for solving a
linear system?

\section{Floating point}
Most floating point numbers are essentially
{\em normalized scientific notation}, but in binary.
A typical normalized number in double precision looks like
\[
  (1.b_1 b_2 b_3 \ldots b_{52})_2 \times 2^{e}
\]
where $b_1 \ldots b_{52}$ are 52 bits of the {\em significand}
that appear after the binary point.  In addition to the normalize
representations, IEEE floating point includes subnormal numbers
(the most important of which is zero) that cannot be represented
in normalized form; $\pm \infty$; and Not-a-Number (NaN), used
to represent the result of operations like $0/0$.

The rule for floating point is that ``basic'' operations
(addition, subtraction, multiplication, division, and square root)
should return the true result, correctly rounded.  So a MATLAB
statement
\begin{lstlisting}
  % Compute the sum of x and y (assuming they are exact)
  z = x + y;
\end{lstlisting}
actually computes $\hat{z} = \operatorname{fl}(x+y)$ where
$\operatorname{fl}(\cdot)$ is the operator that maps real numbers to
the closest floating point representation.  For numbers that are in
the normalized range (i.e.~for which $\operatorname{fl}(z)$ is a
normalized floating point number), the relative error in approximating
$z$ by $\operatorname{fl}(z)$ is smaller in magnitude than machine
epsilon; for double precision, $\epsilon_{\mathrm{mach}} = 2^{-53}
\approx 1.1 \times 10^{-16}$; that is,
\[
  \hat{z} = z(1+\delta), \quad |\delta| \leq \epsilon_{\mathrm{mach}}.
\]
We can {\em model} the effects of roundoff on a computation by writing
a separate $\delta$ term for each arithmetic operation in MATLAB;
this is both incomplete (because it doesn't handle non-normalized
numbers properly) and imprecise (because there is more structure to
the errors than just the bound of machine epsilon).  Nonetheless,
this is a useful way to reason about roundoff when such reasoning
is needed.

\section{Sensitivity, conditioning, and types of error}

In almost every sort of numerical computation, we need to think about
errors.  Errors in numerical computations can come from many different
sources, including:
\begin{itemize}
\item {\em Roundoff error} from inexact computer arithmetic.
\item {\em Truncation error} from approximate formulas.
\item {\em Termination of iterations}.
\item {\em Statistical error}.
\end{itemize}
There are also {\em model errors} that are related not to how
accurately we solve a problem on the computer, but to how accurately
the problem we solve models the state of the world.

There are also several different ways we can think about errors.  The
most obvious is the {\em forward error}: how close is our approximate
answer to the correct answer?  One can also look at {\em backward
  error}: what is the smallest perturbation to the problem such that
our approximation is the true answer?  Or there is {\em residual
  error}: how much do we fail to satisfy the defining equations?

For each type of error, we have to decide whether we want to look at
the {\em absolute} error or the {\em relative} error.  For vector
quantities, we generally want the {\em normwise} absolute or relative
error, but often it's critical to choose norms wisely.  The
{\em condition number} for a problem is the relation between relative
errors in the input (e.g. the right hand side in a linear system of
equations) and relative errors in the output (e.g. the solution to a
linear system of equations).  Typically, we analyze the effect of
roundoff on numerical methods by showing that the method in floating
point is {\em backward stable} (i.e.~the effect of roundoffs lead to
an error that is bounded by some polynomial in the problem size
times $\macheps$) and separately trying to show that the problem is
{\em well-conditioned} (i.e. small backward error in the problem inputs
translates to small forward error in the problem outputs).

We are often concerned with {\em first-order} error analysis,
i.e.~error analysis based on a linearized approximation to the true
problem.  First-order analysis is often adequate to understand the
effect of roundoff error or truncation of certain approximations.
It may not always be enough to understand the effect of large
statistical fluctuations.

\section{Problems}

\begin{enumerate}
\item
  Consider the mapping from quadratic polynomials to cubic polynomials
  given by $p(x) \mapsto x p(x)$.  With respect to the power basis
  $\{1, x, x^2, x^3\}$, what is the matrix associated with this
  mapping?
\item
  Consider the mapping from functions of the form
  $f(x,y) = c_1 + c_2 x + c_3 y$
  to values at $(x_1,y_1)$, $(x_2,y_2)$, and $(x_3,y_3)$.
  What is the associated matrix?  How would you set up a system
  of equations to compute the coefficient vector $c$ associated
  with a vector $b$ of function values at the three points?
\item
  Consider the $L^2$ inner product between quadratic polynomials
  on the interval $[-1,1]$:
  \[
    \langle p, q \rangle = \int_{-1}^1 p(x) q(x) \, dx
  \]
  If we write the polynomials in terms of the power basis
  $\{1, x, x^2\}$, what is the matrix associated with this inner
  product (i.e. the matrix $A$ such that $c_p^T A c_q = \langle p, q
  \rangle$ where $c_p$ and $c_q$ are the coefficient vectors for
  the two polynomials.
\item
  Consider the weighted max norm
  \[
    \|x\| = \max_{j} w_j |x_j|
  \]
  where $w_1, \ldots, w_n$ are positive weights.  For a square matrix
  $A$, what is the operator norm associated with this vector norm?
\item
  If $A$ is symmetric and positive definite, argue that the
  eigendecomposition is the same as the singular value decomposition.
\item
  Consider the block matrix
  \[
    M = \begin{bmatrix} A & B \\ B^T & D \end{bmatrix}
  \]
  where $A$ and $D$ are symmetric and positive definite.  Show that if
  \[
    \lambda_{\min}(A) \lambda_{\min}(D) \geq \|B\|_2^2
  \]
  then the matrix $M$ is symmetric and positive definite.
\item
  Suppose $D$ is a diagonal matrix such that $AD = DA$.  If
  $a_{ij} \neq 0$ for $i \neq j$, what can we say about $D$?
\item
  Convince yourself that the product of two upper triangular
  matrices is itself upper triangular.
\item
  Suppose $Q$ is a differentiable {\em orthogonal} matrix-valued
  function.  Show that $\delta Q = Q S$ where $S$ is skew-symmetric,
  i.e. $S = -S^T$.
\item
  Suppose $Ax = b$ and $(A+D) y = b$ where $A$ is invertible and $D$
  is relatively small.  Assuming we have a fast way to solve systems
  with $A$, give an algorithm to compute $y$ to within an error of
  $O(\|D\|^2)$ in terms of two linear systems involving $A$ and a
  diagonal scaling operation.
\item
  Suppose $r = b-A\hat{x}$ is the residual associated with an
  approximate solution $\hat{x}$.  The {\em maximum componentwise
    relative residual} is
  \[
    \max_i |r_i|/|b_i|.
  \]
  How can this be written in terms of a norm?
\end{enumerate}


\end{document}
