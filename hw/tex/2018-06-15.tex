\documentclass[12pt, leqno]{article} %% use to set typesize
\usepackage{fancyhdr}
\usepackage[sort&compress]{natbib}
\usepackage[letterpaper=true,colorlinks=true,linkcolor=black]{hyperref}

\usepackage{amsfonts}
\usepackage{amsmath}
\usepackage{amssymb}
\usepackage{color}
\usepackage{tikz}
\usepackage{pgfplots}
\usepackage{listings}
%\usepackage{courier}
%\usepackage[utf8]{inputenc}
%\usepackage[russian]{babel}

\lstset{
  numbers=left,
  basicstyle=\ttfamily\footnotesize,
  numberstyle=\tiny\color{gray},
  stepnumber=1,
  numbersep=10pt,
}

\newcommand{\iu}{\ensuremath{\mathrm{i}}}
\newcommand{\bbR}{\mathbb{R}}
\newcommand{\bbC}{\mathbb{C}}
\newcommand{\calV}{\mathcal{V}}
\newcommand{\calW}{\mathcal{W}}
\newcommand{\macheps}{\epsilon_{\mathrm{mach}}}
\newcommand{\matlab}{\textsc{Matlab}}

\newcommand{\ddiag}{\operatorname{diag}}
\newcommand{\fl}{\operatorname{fl}}
\newcommand{\nnz}{\operatorname{nnz}}
\newcommand{\tr}{\operatorname{tr}}
\renewcommand{\vec}{\operatorname{vec}}

\newcommand{\vertiii}[1]{{\left\vert\kern-0.25ex\left\vert\kern-0.25ex\left\vert #1
    \right\vert\kern-0.25ex\right\vert\kern-0.25ex\right\vert}}
\newcommand{\ip}[2]{\langle #1, #2 \rangle}
\newcommand{\ipx}[2]{\left\langle #1, #2 \right\rangle}
\newcommand{\order}[1]{O( #1 )}

\newcommand{\kron}{\otimes}


\newcommand{\hdr}[1]{
  \pagestyle{fancy}
  \lhead{Bindel, Summer 2018}
  \rhead{Numerics for Data Science}
  \fancyfoot{}
  \begin{center}
    {\large{\bf #1}}
  \end{center}
  \lstset{language=matlab,columns=flexible}  
}


\begin{document}
\hdr{2018-06-15}{2018-06-25}

\paragraph*{1: Trying SGD}
Code the basic stochastic gradient algorithm with a (small) fixed step
size for the cricket chirp least squares fit from the June 13 lecture.
Draw two plots:
\begin{itemize}
\item Starting from a zero initial guess, plot the optimality gap (the
  true smallest squared residual vs.~the SGD squared residual)
  against the number of steps.
\item For different values of $\alpha$, run SGD for $10^5$ steps
  starting from the true solution and show a histogram of the
  optimality gap.  Also plot the mean optimality gap as a function of
  $\alpha$; ideally, the two should be proportional.
\end{itemize}

\paragraph*{2: Randomized scaling}
For $A \in \bbR^{m \times n}$, consider the iteration
\[
  x^{k+1} = x^k - \hat{G}^{-1} A^T (Ax^k - b).
\]
Here, $\hat{G}$ is an unbiased estimator for the Gram matrix $A^T A$,
defined by randomly choosing a subset of at least $n$ indices
$\mathcal{I}$:
\[
  \hat{G} =
  \frac{m}{|\mathcal{I}|} \sum_{i \in \mathcal{I}} A_{i,:}^T A_{i,:}
\]
where $A_{i,:}$ representing row $i$ of $A$.  Define
$\rho(\hat{G}) = \|I-\alpha_{\mathrm{opt}} \hat{G}^{-1} A^T A\|$ to be
the rate of convergence for gradient descent with an optimal
step size.  For the cricket data least squares problem,
what is $\log \rho(I)$?  Compare to a histogram of
$\log \rho(\hat{G})$ for a few different sample sizes $|\mathcal{I}|$.

\paragraph*{3: Gauss-Newton}
Code the Gauss-Newton iteration to minimize
\[
  \phi(x) = \sum_j \exp(r_j^2)-1, \quad r = Ax-b,
\]
and illustrate the behavior for the cricket data.

\end{document}
